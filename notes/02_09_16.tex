\subsection{02-09-2016}
\subsubsection{Frequency Domain - Fourier Analysis contd.}
Review Quiz solutions
\begin{itemize}
	\item 1 - D
	\item 2 - B
	\item 3 - A
	\item 4 - E (lot of horizontal edges as per the fourier image - mostly because of the waves)
	\item 5 - C
\end{itemize}
\paragraph {Novak image deconvolution example} - filter is already perfectly known and double precision is used, so the fft values aren't 0 at the higher frequencies but very very low value (if you used single precision or discrete values, they would be 0 - making it difficult to invert).\\
As you add even slight noise, the deconvolved image starts getting worse!

\subsubsection{Sampling}
\paragraph{Aliasing} A better way to down-sample is to apply a low-pass filter first on the image. This removes the high frequencies. Thus we need a lower sampling frequency to reconstruct this image (Nyqist-Shannon theorem).
